\documentclass[11pt]{article}
\usepackage[utf8x]{inputenc}
\usepackage[T1]{fontenc}
\usepackage{amsmath, amsfonts, amssymb}

\begin{document}


Eis um exemplo de equação no modo matemático textual: $x+y$, onde $x$ e $y$
são variáveis tais que 
\begin{equation}
    x = y - 2
\end{equation}

\[
    x^3 - \frac{1}{3} x^{-2} = \frac{\ln 4}{g(x)} \sum^n_{i=0} y_i
\]


\begin{equation*}
    H^\prime \quad H\rho \quad a\quad \rho^\prime\quad \textnormal{do} \quad \Omega^\prime
\end{equation*}

\begin{equation*}
b = (\frac{\int x dx}{1-\frac{y}{8}})^{-2}
\end{equation*}

\begin{equation*}
b = \left( \cfrac{\int x\, dx}{1-\cfrac{y}{8}} \right)^{-2}
\end{equation*}


\begin{equation*}
\frac{1}{\frac{1}{\frac{1}{1}}}
\end{equation*}

\begin{equation*}
\cfrac{1}{\cfrac{1}{\cfrac{1}{1}}}
\end{equation*}

\begin{equation*}
    \oint_{\textsf{\textit{a}}}^{x_\phi} \sinh \frac{1 - \sqrt{\mu}}{\nabla^2\varpi} \, \textsf{\textit{d}}\varpi
\end{equation*}

\begin{equation*}
    \frac{1}{2} < \left\lfloor \textnormal{mod} \left( \left\lfloor \frac{y}{17} \right\rfloor 2^{-17\left\lfloor x \right\rfloor - \textnormal{mod}\left( \left\lfloor y \right \rfloor ,  17 \right)}, 2 \right) \right\rfloor
\end{equation*}


\begin{equation*}
\cfrac{2}{1 + \cfrac{2}{1 + \cfrac{2}{1 + \cfrac{2}{1}}}}
\end{equation*}


\begin{equation*}
\cfrac{2}{1 + \cfrac{2}{1 + \cfrac{2}{1 + \cfrac{2}{1}}}}
\end{equation*}


\end{document}
