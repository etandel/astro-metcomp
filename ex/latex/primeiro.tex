\documentclass[11pt]{article}
\usepackage[utf8x]{inputenc}
\usepackage[T1]{fontenc}
\usepackage{amsmath, amsfonts, amssymb}

\begin{document}

% Escreva aqui o conteudo do seu documento. Por ex:

Jeder hat das Recht auf Bildung. Die Bildung ist unentgeltlich, zum mindesten der
Grundschulunterricht und die grundlegende Bildung. Der Grundschulunterricht ist
obligatorisch. Fach- und Berufsschulunterricht müssen allgemein verfügbar gemacht
werden, und der Hochschulunterricht muß allen gleichermaßen entsprechend ihren
Fähigkeiten offenstehen.

\par Molto diversa è la classificazione che i grammatici Dardano e Trifone danno alla
``Grammatica italiana con nozioni di linguistica'', separano i testi in due gruppi:
il pragmatico e il letterario. Il primo, con la funzione di raccontare, informare,
descrivere, convincere, si proponendo di avere scopi pratici. Il secondo, diverso dai
testi di caratteristiche pratiche, fa un uso unico della lingua perché ha l'intenzione
di avere un insieme di rapporti col resto dell'universo letterario.

m\oe bius
Ta\c c\H e\l\=\i sky
\O rlov
Rom\=a Urbs
Mag\t{oo}


\par {\bf Capitu} era também mais {\it curiosa}. {\sl As curiosidades
de {\rm Capitu} dão para {\Huge um} Capítulo}. \textbf{\textit{Eram de várias
espécies}}, {\tt explicáveis e inexplicáveis}, {\sf assim úteis como
inúteis}, umas graves, outras frívolas, gostava de saber tudo.


\section{A confissão}
Atirei o pau no gato, mas o gato não morreu

\section{A reação da testemunha}
Dona Chica admirou-se com o berro que o gato deu

\subsection{Questões sem resposta}
\begin{itemize}
    \item Por que o gato não morreu?
    \item Por que a sociedade protetora dos animais\footnote{Tampouco soubemos
          de críticas provenientes do IBAMA} não se manifestou sobre o assunto?
\end{itemize}

\subsubsection{Personagens de um conto trágico}
\begin{enumerate}
    \item O agressor
    \item O gato
    \item Dona Chica
\end{enumerate}


\section{Math}
Eis um exemplo de equação no modo matemático textual: $x+y$, onde $x$ e $y$
são variáveis tais que 
\begin{equation}
%
    x = y - 2
%
\end{equation}


\end{document}
